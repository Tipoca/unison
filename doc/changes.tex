\begin{changesfromversion}{2.40.1}
\item Added "BelowPath" patterns, that match a path as well as all paths below
  (convenient to use with no{deletion,update,creation}partial preferences)
\item Added a "fat" preference that makes Unison use the right options
  when one of the replica is on a FAT filesystem.
\item Allow "prefer/force=newer" even when not synchronizing modification
  times.  (The reconciler will not be aware of the modification time
  of unchanged files, so the synchronization choices of Unison can be
  different from when "times=true", but the behavior remains sane:
  changed files with the most recent modification time will be
  propagated.)
\item Minor fixes and improvements:
\begin{itemize}
\item Compare filenames up to decomposition in case sensitive mode when
  one host is running MacOSX and the unicode preference is set to
  true.
\item Rsync: somewhat faster compressor
\item Make Unicode the default on all architectures (it was only the
  default when a Mac OS X or Windows machine was involved).
\end{itemize}
\end{changesfromversion}

\begin{changesfromversion}{2.32}
\item Major enhancement: Unicode support.  
\begin{itemize}
\item Unison should now handle unicode filenames correctly on all platforms.
\item This functionality is controlled by a new preference {\tt unicode}.
\item Unicode mode is now the default when one of the hosts is under
  Windows or MacOS.  This may make upgrades a bit more painful (the
  archives cannot be reused), but this is a much saner default.
\end{itemize}
\item Partial transfer of directories.  If an error occurs while
  transferring a directory, the part transferred so far is copied into
  place (and the archives are updated accordingly).
  The "maxerrors" preference controls how many transfer error Unison
  will accept before stopping the transfer of a directory (by default,
  only one).  This makes it possible to transfer most of a directory
  even if there are some errors.  Currently, only the first error is
  reported by the GUIs.

  Also, allow partial transfer of a directory when there was an error deep
  inside this directory during update detection.  At the moment, this
  is only activated with the text and GTK UIs, which have been
  modified so that they show that the transfer is going to be partial
  and so that they can display all errors.
\item Improvement to the code for resuming directory transfers:
\begin{itemize}
\item 
   if a file was not correctly transferred (or the source has been
    modified since, with unchanged size), Unison performs a new
    transfer rather than failing
  \item spurious files are deleted (this can happen if a file is deleted
    on the source replica before resuming the transfer; not deleting
    the file would result in it reappearing on the target replica)
\end{itemize}
\item Experimental streaming protocol for transferring file contents (can
  be disabled by setting the directive "stream" to false): file
  contents is transfered asynchronously (without waiting for a response
  from the destination after each chunk sent) rather than using the
  synchronous RPC mechanism.  As a consequence:
  \begin{itemize}
  \item 
   Unison now transfers the contents of a single file at a time
    (Unison used to transfer several contents simultaneously in order
    to hide the connection latency.)
  \item the transfer of large files uses the full available bandwidth
    and is not slowed done due to the connection latency anymore
  \item we get performance improvement for small files as well by
    scheduling many files simultaneously (as scheduling a file for
    transfer consume little ressource: it does not mean allocating a
    large buffer anymore)
  \end{itemize}
\item Changes to the internal implementation of the rsync algorithm:
\begin{itemize}
\item 
  use longer blocks for large files (the size of a block is the
    square root of the size of the file for large files);
  \item transmit less checksum information per block (we still have less
    than one chance in a hundred million of transferring a file
    incorrectly, and Unison will catch any transfer error when
    fingerprinting the whole file)
  \item avoid transfer overhead (which was 4 bytes per block)
\end{itemize}
  For a 1G file, the first optimization saves a factor 50 on the
  amount of data transferred from the target to the source (blocks
  are 32768 bytes rather than just 700 bytes).  The two other
  optimizations save another factor of 2 (from 24 bytes per block
  down to 10).
\item Implemented an on-disk file fingerprint cache to speed-up update
  detection after a crash: this way, Unison does not have do recompute
  all the file fingerprints from scratch.
  \begin{itemize}
  \item When Unison detects that the archive case-sensitivity mode
  does not match the current settings, it populates the fingerprint
  cache using the archive contents.  This way, changing the
  case-sensitivity mode should be reasonably fast.
  \end{itemize}
\item New preferences "noupdate=root", "nodeletion=root", "nocreation=root"
  that prevent Unison from performing files updates, deletions or
  creations on the given root.  Also 'partial' versions of 'noupdate',
  'nodeletion' and 'nocreation' 
\item Limit the number of simultaneous external copy program
  ("copymax" preference)
\item New "links" preference.  When set to false, Unison will report an
  error on symlinks during update detection.  (This is the default
  when one host is running Windows but not Cygwin.)  This is better
  than failing during propagation.
\item Added a preference "halfduplex" to force half-duplex communication
  with the server.  This may be useful on unreliable links (as a more
  efficient alternative to "maxthreads = 1").
\item Renamed preference "pretendwin" to "ignoreinodenumbers" (an alias is
  kept for backwards compatibility).
\item Ignore one-second differences when synchronizing modification time.
  (Technically, this is an incompatible archive format change, but it
   is backward compatible.  To trigger a problem, a user would have to
   synchronize modification times on a filesystem with a two-second
   granularity and then downgrade to a previous version of Unison,
   which does not work well in such a case.  Thus, it does not
   seem worthwhile to increment the archive format number, which would
   impact all users.)
\item Do not keep many files simultaneously opened anymore when the rsync
  algorithm is in use.
\item Add ``ignorearchives'' preference to ignore existing archives (to
  avoid forcing users to delete them manually, in situations where one
  archive has gotten   deleted or corrupted).
\item Mac OS
\begin{itemize}
\item fixed rsync bug which could result in an "index out of bounds"
  error when transferring resource forks.
\item Fixed bug which made Unison ignore finder information and resource
  fork when compiled to 64bit on Mac OSX.
\item should now be 64 bit clean (the Growl framework is not up to date,
    though)
\item Made the bridge between Objective C and Ocaml code GC friendly
    (it was allocating ML values and putting them in an array which
    was not registered with the GC)
\item use darker grey arrows (patch contributed by Eric Y. Kow)
\end{itemize}
\item GTK user interface
\begin{itemize}
\item  assistant for creating profiles
\item profile editor
\item pop up a summary window when the replicas are not fully
    synchronized after transport
\item display estimated remaining time and transfer rate on the
  progress bar
\item allow simultaneous selection of several items
\item Do not reload the preference file before a new update
  detection if it is unchanged
\item disabled scrolling to the first unfinished item during transport.
  It goes way too fast when lot of small files are synchronized, and it
  makes it impossible to browse the file list during transport.
\item take into account the "height" preference again
\item the internal list of selected reconciler item was not always in
    sync with what was displayed (GTK bug?); workaround implemented
\item Do not display "Looking for change" messages during propagation
  (when checking the targe is unchanged) but only during update detection
\item Apply patch to fix some crashes in the OSX GUI, thanks to Onne Gorter.
\end{itemize}
\item Text UI
\begin{itemize}
\item During update detection, display status by updating a single line
rather than generating a new line of output every so often.  Should be less
confusing.
\end{itemize}
\item Windows 
\begin{itemize}
\item Fastcheck is now the default under Windows.  People mostly use NTFS
  nowadays and the Unicode API provides an equivalent to inode numbers
  for this filesystem.
\item Only use long UNC path for accessing replicas (as '..' is
  not handled with this format of paths, but can be useful)
\item Windows text UI: now put the console into UTF-8 output mode.  This
  is the right thing to do when in Unicode mode, and is no worse than
  what we had previously otherwise (the console use some esoteric
  encoding by default).  This only works when using a Unicode font
  instead of the default raster font.
\item Don't get the home directory from environment variable HOME under
  Windows (except for Cygwin binaries): we don't want the behavior of
  Unison to depends on whether it is run from a Cygwin shell (where
  HOME is set) or in any other way (where HOME is usually not set).
\end{itemize}
\item Miscellaneous fixes and improvements
\begin{itemize}
\item Made a server waiting on a socket more resilient to unexpected
  lost connections from the client.
\item Small patch to property setting code suggested by Ulrich Gernkow.
\item Several fixes to the change transfer functions (both the internal ones
  and external transfers using rsync).  In particular, limit the number of
  simultaneous transfer using an rsync
  (as the rsync algorithm can use a large amount of memory when
   processing huge files)
\item Keep track of which file contents are being transferred, and delay
  the transfer of a file when another file with the same contents is
  currently being transferred.  This way, the second transfer can be
  skipped and replaced by a local copy.
\item Experimental update detection optimization:
  do not read the contents of unchanged directories
\item When a file transfer fails, turn off fastcheck for this file on the
  next sync.
\item Fixed bug with case insensitive mode on a case sensitive filesystem:
\begin{itemize}
\item 
   if file "a/a" is created on one replica and directory "A" is
    created on the other, the file failed to be synchronized the first
    time Unison is run afterwards, as Unison uses the wrong path "a/a"
    (if Unison is run again, the directories are in the archive, so
     the right path is used);
  \item if file "a" appears on one replica and file "A" appears on the
    other with different contents, Unison was unable to synchronize
    them.
\end{itemize}
\item Improved error reporting when the destination is updated during
  synchronization: Unison now tells which file has been updated, and how.
\item Limit the length of temporary file names
\item Case sensitivity information put in the archive (in a backward
  compatible way) and checked when the archive is loaded
\item Got rid of the 16mb marshalling limit by marshalling to a bigarray.
\item Resume copy of partially transferred files.
\end{itemize}
\end{changesfromversion}

\begin{changesfromversion}{2.31}
\item Small user interface changes
\begin{itemize}
\item Small change to text UI "scanning..." messages, to print just
  directories (hopefully making it clearer that individual files are
  not necessarily being fingerprinted).  
\end{itemize}
\item Minor fixes and improvements:
\begin{itemize}
\item Ignore one hour differences when deciding whether a file may have
  been updated.  This avoids slow update detection after daylight
  saving time changes under Windows.  This makes Unison slightly more
  likely to miss an update, but it should be safe enough.
\item Fix a small bug that was affecting mainly windows users.  We need to
  commit the archives at the end of the sync even if there are no
  updates to propagate because some files (in fact, if we've just
  switched to DST on windows, a LOT of files) might have new modtimes
  in the archive.  (Changed the text UI only.  It's less clear where
  to change the GUI.)
\item Don't delete the temp file when a transfer fails due to a
  fingerprint mismatch (so that we can have a look and see why!)  We've also
  added more debugging code togive more informative error messages when we
  encounter the dreaded and longstanding "assert failed during file
  transfer" bug
\item Incorrect paths ("path" directive) now result in an error update
  item rather than a fatal error.
\item Create parent directories (with correct permissions) during
  transport for paths which point to non-existent locations in the
  destination replica.
\end{itemize}
\end{changesfromversion}

\begin{changesfromversion}{2.27}
\item If Unison is interrupted during a directory transfer, it will now
leave the partially transferred directory intact in a temporary
location. (This maintains the invariant that new files/directories are
transferred either completely or not at all.)  The next time Unison is run,
it will continue filling in this temporary directory, skipping transferring
files that it finds are already there.
\item We've added experimental support for invoking an external file
transfer tool for whole-file copies instead of Unison's built-in transfer
protocol.  Three new preferences have been added:
\begin{itemize}
\item {\tt copyprog} is a string giving the name (and command-line
switches, if needed) of an external program that can be used to copy large
files efficiently.  By default, rsync is invoked, but other tools such as
scp can be used instead by changing the value of this preference.  (Although
this is not its primary purpose, rsync is actually a pretty fast way of
copying files that don't already exist on the receiving host.)  For files
that do already exist on (but that have been changed in one replica), Unison
will always use its built-in implementation of the rsync algorithm.
\item Added a "copyprogrest" preference, so that we can give different
command lines for invoking the external copy utility depending on whether a
partially transferred file already exists or not.  (Rsync doesn't seem to
care about this, but other utilities may.)
\item {\tt copythreshold} is an integer (-1 by default), indicating above what
filesize (in megabytes) Unison should use the external copying utility
specified by copyprog.  Specifying 0 will cause ALL copies to use the
external program; a negative number will prevent any files from using it.
(Default is -1.)
\end{itemize}
Thanks to Alan Schmitt for a huge amount of hacking and to an anonymous
sponsor for suggesting and underwriting this extension.
\item Small improvements:
\begin{itemize}
\item Added a new preference, {\tt dontchmod}.  By default, Unison uses the
{\tt chmod} system call to set the permission bits of files after it has
copied them.  But in some circumstances (and under some operating systems),
the chmod call always fails.  Setting this preference completely prevents
Unison from ever calling {\tt chmod}.
\item Don't ignore files that look like backup files if the {\tt
  backuplocation} preference is set to {\tt central}
\item Shortened the names of several preferences.  The old names are also
still supported, for backwards compatibility, but they do not appear in the
documentation.
\item Lots of little documentation tidying.  (In particular, preferences are
separated into Basic and Advanced!  This should hopefully make Unison a
little more approachable for new users.
\item Unison can sometimes fail to transfer a file, giving the unhelpful
message "Destination updated during synchronization" even though the file
has not been changed.  This can be caused by programs that change either the
file's contents \emph{or} the file's extended attributes without changing
its modification time.  It's not clear what is the best fix for this -- it
is not Unison's fault, but it makes Unison's behavior puzzling -- but at
least Unison can be more helpful about suggesting a workaround (running once
with {\tt fastcheck} set to false).  The failure message has been changed to
give this advice.
\item Further improvements to the OS X GUI (thanks to Alan Schmitt and Craig
Federighi).
\end{itemize}
\item Very preliminary support for triggering Unison from an external 
  filesystem-watching utility.  The current implementation is very
  simple, not efficient, and almost completely untested---not ready 
  for real users.  But if someone wants to help improve it (e.g.,
  by writing a filesystem watcher for your favorite OS), please make
  yourself known!

  On the Unison side, the new behavior is very simple:
  \begin{itemize}
  \item use the text UI 
    \item start Unison with the command-line flag "-repeat FOO", 
      where FOO is name of a file where Unison should look 
      for notifications of changes
    \item  when it starts up, Unison will read the whole contents 
      of this file (on both hosts), which should be a 
      newline-separated list of paths (relative to the root 
      of the synchronization) and synchronize just these paths, 
      as if it had been started with the "-path=xxx" option for 
      each one of them
    \item when it finishes, it will sleep for a few seconds and then
      examine the watchfile again; if anything has been added, it
      will read the new paths, synchronize them, and go back to 
      sleep
    \item that's it!
  \end{itemize}
  To use this to drive Unison "incrementally," just start it in 
  this mode and start up a tool (on each host) to watch for
  new changes to the filesystem and append the appropriate paths
  to the watchfile.  Hopefully such tools should not be too hard
  to write.
\item Bug fixes:
\begin{itemize}
\item Fixed a bug that was causing new files to be created with
  permissions 0x600 instead of using a reasonable default (like
  0x644), if the 'perms' flag was set to 0.  (Bug reported by Ben
  Crowell.)
\item Follow maxthreads preference when transferring directories.
\end{itemize}
\end{changesfromversion}

\begin{changesfromversion}{2.17}
\item Major rewrite and cleanup of the whole Mac OS X graphical user
interface by Craig Federighi.  Thanks, Craig!!!
\item Small fix to ctime (non-)handling in update detection under windows
  with fastcheck.  
\item Several small fixes to the GTK2 UI to make it work better under
Windows [thanks to Karl M for these].
\item The backup functionality has been completely rewritten.  The external
interface has not changed, but numerous bugs, irregular behaviors, and
cross-platform inconsistencies have been corrected.
\item The Unison project now accepts donations via PayPal.  If you'd like to
donate, you can find a link to the donation page on the
\URL{http://www.cis.upenn.edu/~bcpierce/unison/lists.html}{Unison home
  page}.
\item Some important safety improvements:
\begin{itemize}
\item Added a new \verb|mountpoint| preference, which can be used to specify
a path that must exist in both replicas at the end of update detection
(otherwise Unison aborts).  This can be used to avoid potentially dangerous
situations when Unison is used with removable media such as external hard
drives and compact flash cards.  
\item The confirmation of ``big deletes'' is now controlled by a boolean preference
  \verb|confirmbigdeletes|.  Default is true, which gives the same behavior as
  previously.  (This functionality is at least partly superceded by the
  \verb|mountpoint| preference, but it has been left in place in case it is
  useful to some people.)
  \item If Unison is asked to ``follow'' a symbolic link but there is
  nothing at the other end of the link, it will now flag this path as an
  error, rather than treating the symlink itself as missing or deleted.
  This avoids a potentially dangerous situation where a followed symlink
  points to an external filesystem that might be offline when Unison is run
  (whereupon Unison would cheerfully delete the corresponding files in the
  other replica!).
\end{itemize}

\item Smaller changes:
\begin{itemize}
\item Added \verb|forcepartial| and \verb|preferpartial| preferences, which
behave like \verb|force| and \verb|prefer| but can be specified on a
per-path basis. [Thanks to Alan Schmitt for this.]
\item A bare-bones self test feature was added, which runs unison through
  some of its paces and checks that the results are as expected.  The
  coverage of the tests is still very limited, but the facility has already
  been very useful in debugging the new backup functionality (especially in
  exposing some subtle cross-platform issues).
\item Refined debugging code so that the verbosity of individual modules
  can be controlled separately.  Instead of just putting '-debug
  verbose' on the command line, you can put '-debug update+', which
  causes all the extra messages in the Update module, but not other
  modules, to be printed.  Putting '-debug verbose' causes all modules
  to print with maximum verbosity.
\item Removed \verb|mergebatch| preference.  (It never seemed very useful, and
  its semantics were confusing.)
\item Rewrote some of the merging functionality, for better cooperation
  with external Harmony instances.
\item Changed the temp file prefix from \verb|.#| to \verb|.unison|.
\item Compressed the output from the text user interface (particularly
  when run with the \verb|-terse| flag) to make it easier to interpret the
  results when Unison is run several times in succession from a script.
\item Diff and merge functions now work under Windows.
\item Changed the order of arguments to the default diff command (so that
  the + and - annotations in diff's output are reversed).
\item Added \verb|.mpp| files to the ``never fastcheck'' list (like
\verb|.xls| files). 
\end{itemize}

\item Many small bugfixes, including:
\begin{itemize}
\item Fixed a longstanding bug regarding fastcheck and daylight saving time
  under Windows when Unison is set up to synchronize modification times.
  (Modification times cannot be updated in the archive in this case,
  so we have to ignore one hour differences.)
\item Fixed a bug that would occasionally cause the archives to be left in
  non-identical states on the two hosts after synchronization.
\item Fixed a bug that prevented Unison from communicating correctly between
  32- and 64-bit architectures.
\item On windows, file creation times are no longer used as a proxy for
  inode numbers.  (This is unfortunate, as it makes fastcheck a little less
  safe.  But it turns out that file creation times are not reliable 
  under Windows: if a file is removed and a new file is created in its
  place, the new one will sometimes be given the same creation date as the
  old one!)
\item Set read-only file to R/W on OSX before attempting to change other attributes.
\item Fixed bug resulting in spurious "Aborted" errors during transport
(thanks to Jerome Vouillon) 
\item Enable diff if file contents have changed in one replica, but
only properties in the other.
\item Removed misleading documentation for 'repeat' preference.
\item Fixed a bug in merging code where Unison could sometimes deadlock
  with the external merge program, if the latter produced large
  amounts of output.
\item Workaround for a bug compiling gtk2 user interface against current versions
  of gtk2+ libraries.  
\item Added a better error message for "ambiguous paths".
\item Squashed a longstanding bug that would cause file transfer to fail
  with the message ``Failed: Error in readWrite: Is a directory.''
\item Replaced symlinks with copies of their targets in the Growl framework in src/uimac.
  This should make the sources easier to check out from the svn repository on WinXP
  systems.
\item Added a workaround (suggested by Karl M.) for the problem discussed
  on the unison users mailing list where, on the Windows platform, the
  server would hang when transferring files.  I conjecture that
  the problem has to do with the RPC mechanism, which was used to
  make a call {\em back} from the server to the client (inside the Trace.log 
  function) so that the log message would be appended to the log file on 
  the client.  The workaround is to dump these messages (about when
  xferbycopying shortcuts are applied and whether they succeed) just to the
  standard output of the Unison process, not to the log file.
\end{itemize}
\end{changesfromversion}

\begin{changesfromversion}{2.13.0}
\item The features for performing backups and for invoking external merge
programs have been completely rewritten by Stephane Lescuyer (thanks,
Stephane!).  The user-visible functionality should not change, but the
internals have been rationalized and there are a number of new features.
See the manual (in particular, the description of the \verb|backupXXX|
preferences) for details.
\item Incorporated patches for ipv6 support, contributed by Samuel Thibault.
(Note that, due to a bug in the released OCaml 3.08.3 compiler, this code
will not actually work with ipv6 unless compiled with the CVS version of the
OCaml compiler, where the bug has been fixed; however, ipv4 should continue
to work normally.)
\item OSX interface:
\begin{itemize}
\item Incorporated Ben Willmore's cool new icon for the Mac UI.
\end{itemize}
\item Small fixes:
\begin{itemize}
\item Fixed off by one error in month numbers (in printed dates) reported 
  by Bob Burger
\end{itemize}
\end{changesfromversion}

\begin{changesfromversion}{2.12.0}
\item New convention for release numbering: Releases will continue to be
given numbers of the form \verb|X.Y.Z|, but, 
from now on, just the major version number (\verb|X.Y|) will be considered
significant when checking compatibility between client and server versions.
The third component of the version number will be used only to identify
``patch levels'' of releases.

This change goes hand in hand with a change to the procedure for making new
releases.  Candidate releases will initially be given ``beta release''
status when they are announced for public consumption.  Any bugs that are
discovered will be fixed in a separate branch of the source repository
(without changing the major version number) and new tarballs re-released as
needed.  When this process converges, the patched beta version will be
dubbed stable.
\item Warning (failure in batch mode) when one path is completely emptied.
  This prevents Unison from deleting everything on one replica when
  the other disappear.
\item Fix diff bug (where no difference is shown the first time the diff
  command is given).
\item User interface changes:
\begin{itemize}
\item Improved workaround for button focus problem (GTK2 UI)
\item Put leading zeroes in date fields
\item More robust handling of character encodings in GTK2 UI
\item Changed format of modification time displays, from \verb|modified at hh:mm:ss on dd MMM, yyyy|
to \verb|modified on yyyy-mm-dd hh:mm:ss|
\item Changed time display to include seconds (so that people on FAT
  filesystems will not be confused when Unison tries to update a file
  time to an odd number of seconds and the filesystem truncates it to
  an even number!)
\item Use the diff "-u" option by default when showing differences between files
  (the output is more readable)
\item In text mode, pipe the diff output to a pager if the environment
  variable PAGER is set
\item Bug fixes and cleanups in ssh password prompting.  Now works with
  the GTK2 UI under Linux.  (Hopefully the Mac OS X one is not broken!)
\item Include profile name in the GTK2 window name
\item Added bindings ',' (same as '<') and '.' (same as '>') in the GTK2 UI
\end{itemize}
\item Mac GUI:
\begin{itemize}
\item actions like < and > scroll to the next item as necessary.
\item Restart has a menu item and keyboard shortcut (command-R).
\item 
    Added a command-line tool for Mac OS X.  It can be installed from
    the Unison menu.
\item New icon.
\item   Handle the "help" command-line argument properly.
\item   Handle profiles given on the command line properly.
\item  When a profile has been selected, the profile dialog is replaced by a
    "connecting" message while the connection is being made.  This
    gives better feedback.
\item   Size of left and right columns is now large enough so that
    "PropsChanged" is not cut off.
\end{itemize}
\item Minor changes:
\begin{itemize}
\item Disable multi-threading when both roots are local
\item Improved error handling code.  In particular, make sure all files
  are closed in case of a transient failure
\item Under Windows, use \verb|$UNISON| for home directory as a last resort
  (it was wrongly moved before \verb|$HOME| and \verb|$USERPROFILE| in
  Unison 2.12.0)
\item Reopen the logfile if its name changes (profile change)
\item Double-check that permissions and modification times have been
  properly set: there are some combination of OS and filesystem on
  which setting them can fail in a silent way.
\item Check for bad Windows filenames for pure Windows synchronization
  also (not just cross architecture synchronization).
  This way, filenames containing backslashes, which are not correctly
  handled by unison, are rejected right away.
\item Attempt to resolve issues with synchronizing modification times
  of read-only files under Windows
\item Ignore chmod failures when deleting files
\item Ignore trailing dots in filenames in case insensitive mode
\item Proper quoting of paths, files and extensions ignored using the UI
\item The strings CURRENT1 and CURRENT2 are now correctly substitued when
  they occur in the diff preference
\item Improvements to syncing resource forks between Macs via a non-Mac system.
\end{itemize}
\end{changesfromversion}

\begin{changesfromversion}{2.10.2}
\item \incompatible{} Archive format has changed.  
\item Source code availability: The Unison sources are now managed using
  Subversion.  One nice side-effect is that anonymous checkout is now
  possible, like this:
\begin{verbatim}
        svn co https://cvs.cis.upenn.edu:3690/svnroot/unison/
\end{verbatim}
We will also continue to export a ``developer tarball'' of the current
(modulo one day) sources in the web export directory.  To receive commit logs
for changes to the sources, subscribe to the \verb|unison-hackers| list
(\ONEURL{http://www.cis.upenn.edu/~bcpierce/unison/lists.html}). 
\item Text user interface:
\begin{itemize}
\item Substantial reworking of the internal logic of the text UI to make it
a bit easier to modify.
\item The {\tt dumbtty} flag in the text UI is automatically set to true if
the client is running on a Unix system and the {\tt EMACS} environment
variable is set to anything other than the empty string.
\end{itemize}
\item Native OS X gui:
\begin{itemize}
\item Added a synchronize menu item with keyboard shortcut
\item Added a merge menu item, still needs to be debugged
\item Fixes to compile for Panther
\item Miscellaneous improvements and bugfixes
\end{itemize}
\item Small changes:
\begin{itemize}
\item Changed the filename checking code to apply to Windows only, instead
  of OS X as well.
\item Finder flags now synchronized
\item Fallback in copy.ml for filesystem that do not support \verb|O_EXCL|
\item  Changed buffer size for local file copy (was highly inefficient with
  synchronous writes)
\item Ignore chmod failure when deleting a directory
\item  Fixed assertion failure when resolving a conflict content change /
  permission changes in favor of the content change.
\item Workaround for transferring large files using rsync.
\item Use buffered I/O for files (this is the only way to open files in binary
  mode under Cygwin).
\item On non-Cygwin Windows systems, the UNISON environment variable is now checked first to determine 
  where to look for Unison's archive and preference files, followed by \verb|HOME| and 
  \verb|USERPROFILE| in that order.  On Unix and Cygwin systems, \verb|HOME| is used.
\item Generalized \verb|diff| preference so that it can be given either as just 
  the command name to be used for calculating diffs or else a whole command
  line, containing the strings \verb|CURRENT1| and \verb|CURRENT2|, which will be replaced
  by the names of the files to be diff'ed before the command is called.
\item Recognize password prompts in some newer versions of ssh.
\end{itemize}
\end{changesfromversion}

\begin{changesfromversion}{2.9.20}
\item \incompatible{} Archive format has changed.  
\item Major functionality changes:
\begin{itemize}
\item Major tidying and enhancement of 'merge' functionality.  The main
  user-visible change is that the external merge program may either write
  the merged output to a single new file, as before, or it may modify one or
  both of its input files, or it may write {\em two} new files.  In the
  latter cases, its modifications will be copied back into place on both the
  local and the remote host, and (if the two files are now equal) the
  archive will be updated appropriately.  More information can be found in
  the user manual.  Thanks to Malo Denielou and Alan Schmitt for these
  improvements.

  Warning: the new merging functionality is not completely compatible with
  old versions!  Check the manual for details.
\item Files larger than 2Gb are now supported.
\item Added preliminary (and still somewhat experimental) support for the
  Apple OS X operating system.   
\begin{itemize}
\item Resource forks should be transferred correctly.  (See the manual for
details of how this works when synchronizing HFS with non-HFS volumes.)
Synchronization of file type and creator information is also supported.
\item On OSX systems, the name of the directory for storing Unison's
archives, preference files, etc., is now determined as follows:
\begin{itemize}
    \item if \verb+~/.unison+ exists, use it
     \item otherwise, use \verb|~/Library/Application Support/Unison|, 
         creating it if necessary.
\end{itemize}
\item A preliminary native-Cocoa user interface is under construction.  This
still needs some work, and some users experience unpredictable crashes, so
it is only for hackers for now.  Run make with {\tt UISTYLE=mac} to build
this interface.
\end{itemize}
\end{itemize}

\item Minor functionality changes:
\begin{itemize}
\item Added an {\tt ignorelocks} preference, which forces Unison to override left-over
  archive locks.  (Setting this preference is dangerous!  Use it only if you
  are positive you know what you are doing.) 
% BCP: removed later
% \item Running with the {\tt -timers} flag set to true will now show the total time taken
%   to check for updates on each directory.  (This can be helpful for tidying directories to improve
%   update detection times.)  
\item Added a new preference {\tt assumeContentsAreImmutable}.  If a directory
  matches one of the patterns set in this preference, then update detection
  is skipped for files in this directory.  (The 
  purpose is to speed update detection for cases like Mail folders, which
  contain lots and lots of immutable files.)  Also a preference
  {\tt assumeContentsAreImmutableNot}, which overrides the first, similarly
  to {\tt ignorenot}.  (Later amendment: these preferences are now called
  {\tt immutable} and {\tt immutablenot}.)
\item The {\tt ignorecase} flag has been changed from a boolean to a three-valued
  preference.  The default setting, called {\tt default}, checks the operating systems
  running on the client and server and ignores filename case if either of them is
  OSX or Windows.  Setting ignorecase to {\tt true} or {\tt false} overrides
  this behavior.  If you have been setting {\tt ignorecase} on the command
  line using {\tt -ignorecase=true} or {\tt -ignorecase=false}, you will
  need to change to {\tt -ignorecase true} or {\tt -ignorecase false}.
\item a new preference, 'repeat', for the text user interface (only).  If 'repeat' is set to
  a number, then, after it finishes synchronizing, Unison will wait for that many seconds and
  then start over, continuing this way until it is killed from outside.  Setting repeat to true
  will automatically set the batch preference to true.  
\item Excel files are now handled specially, so that the {\tt fastcheck}
  optimization is skipped even if the {\tt fastcheck} flag is set.  (Excel
  does some naughty things with modtimes, making this optimization
  unreliable and leading to failures during change propagation.)
\item The ignorecase flag has been changed from a boolean to a three-valued
  preference.  The default setting, called 'default', checks the operating systems
  running on the client and server and ignores filename case if either of them is
  OSX or Windows.  Setting ignorecase to 'true' or 'false' overrides this behavior.
\item Added a new preference, 'repeat', for the text user interface (only,
  at the moment).  If 'repeat' is set to a number, then, after it finishes
  synchronizing, Unison will wait for that many seconds and then start over,
  continuing this way until it is killed from outside.  Setting repeat to
  true will automatically set the batch preference to true.
\item The 'rshargs' preference has been split into 'rshargs' and 'sshargs' 
  (mainly to make the documentation clearer).  In fact, 'rshargs' is no longer
  mentioned in the documentation at all, since pretty much everybody uses
  ssh now anyway.
\end{itemize}
\item Documentation
\begin{itemize}
\item The web pages have been completely redesigned and reorganized.
  (Thanks to Alan Schmitt for help with this.)
\end{itemize}
\item User interface improvements
\begin{itemize}
\item Added a GTK2 user interface, capable (among other things) of displaying filenames
  in any locale encoding.  Kudos to Stephen Tse for contributing this code!  
\item The text UI now prints a list of failed and skipped transfers at the end of
  synchronization. 
\item Restarting update detection from the graphical UI will reload the current
  profile (which in particular will reset the -path preference, in case
  it has been narrowed by using the ``Recheck unsynchronized items''
  command).
\item Several small improvements to the text user interface, including a
  progress display.
\end{itemize}
\item Bug fixes (too numerous to count, actually, but here are some):
\begin{itemize}
\item The {\tt maxthreads} preference works now.
\item Fixed bug where warning message about uname returning an unrecognized
  result was preventing connection to server.  (The warning is no longer
  printed, and all systems where 'uname' returns anything other than 'Darwin' 
  are assumed not to be running OS X.)
\item Fixed a problem on OS X that caused some valid file names (e.g.,
  those including colons) to be considered invalid.
\item Patched Path.followLink to follow links under cygwin in addition to Unix
  (suggested by Matt Swift).
\item Small change to the storeRootsName function, suggested by bliviero at 
  ichips.intel.com, to fix a problem in unison with the `rootalias'
  option, which allows you to tell unison that two roots contain the same 
  files.  Rootalias was being applied after the hosts were 
  sorted, so it wouldn't work properly in all cases.
\item Incorporated a fix by Dmitry Bely for setting utimes of read-only files
  on Win32 systems.   
\end{itemize}
\item Installation / portability:
\begin{itemize}
\item Unison now compiles with OCaml version 3.07 and later out of the box.
\item Makefile.OCaml fixed to compile out of the box under OpenBSD.
\item a few additional ports (e.g. OpenBSD, Zaurus/IPAQ) are now mentioned in 
  the documentation 
\item Unison can now be installed easily on OSX systems using the Fink
  package manager
\end{itemize}
\end{changesfromversion}

\begin{changesfromversion}{2.9.1}
\item Added a preference {\tt maxthreads} that can be used to limit the
number of simultaneous file transfers.
\item Added a {\tt backupdir} preference, which controls where backup
files are stored.
\item Basic support added for OSX.  In particular, Unison now recognizes
when one of the hosts being synchronized is running OSX and switches to
a case-insensitive treatment of filenames (i.e., 'foo' and 'FOO' are
considered to be the same file).
  (OSX is not yet fully working,
  however: in particular, files with resource forks will not be
  synchronized correctly.)
\item The same hash used to form the archive name is now also added to
the names of the temp files created during file transfer.  The reason for
this is that, during update detection, we are going to silently delete
any old temp files that we find along the way, and we want to prevent
ourselves from deleting temp files belonging to other instances of Unison
that may be running in parallel, e.g. synchronizing with a different
host.  Thanks to Ruslan Ermilov for this suggestion.
\item Several small user interface improvements
\item Documentation
\begin{itemize}
\item FAQ and bug reporting instructions have been split out as separate
      HTML pages, accessible directly from the unison web page.
\item Additions to FAQ, in particular suggestions about performance
tuning. 
\end{itemize}
\item Makefile
\begin{itemize}
\item Makefile.OCaml now sets UISTYLE=text or UISTYLE=gtk automatically,
  depending on whether it finds lablgtk installed
\item Unison should now compile ``out of the box'' under OSX
\end{itemize}
\end{changesfromversion}

\begin{changesfromversion}{2.8.1}
\item Changing profile works again under Windows
\item File movement optimization: Unison now tries to use local copy instead of
  transfer for moved or copied files.  It is controled by a boolean option
  ``xferbycopying''.
\item Network statistics window (transfer rate, amount of data transferred).
      [NB: not available in Windows-Cygwin version.]
\item symlinks work under the cygwin version (which is dynamically linked).
\item Fixed potential deadlock when synchronizing between Windows and
Unix 
\item Small improvements:
  \begin{itemize} 
  \item If neither the {\tt USERPROFILE} nor the {\tt HOME} environment
    variables are set, then Unison will put its temporary commit log
    (called {\tt DANGER.README}) into the directory named by the 
    {\tt UNISON} environment variable, if any; otherwise it will use
    {\tt C:}.
  \item alternative set of values for fastcheck: yes = true; no = false;
  default = auto.
  \item -silent implies -contactquietly
  \end{itemize}
\item Source code:
  \begin{itemize}
  \item Code reorganization and tidying.  (Started breaking up some of the
    basic utility modules so that the non-unison-specific stuff can be
    made available for other projects.)
  \item several Makefile and docs changes (for release);
  \item further comments in ``update.ml'';
  \item connection information is not stored in global variables anymore.
  \end{itemize}
\end{changesfromversion}

\begin{changesfromversion}{2.7.78}
\item Small bugfix to textual user interface under Unix (to avoid leaving
  the terminal in a bad state where it would not echo inputs after Unison
  exited).
\end{changesfromversion}

\begin{changesfromversion}{2.7.39}
\item Improvements to the main web page (stable and beta version docs are
  now both accessible).
\item User manual revised.
\item Added some new preferences:
\begin{itemize}
\item ``sshcmd'' and ``rshcmd'' for specifying paths to ssh and rsh programs.
\item ``contactquietly'' for suppressing the ``contacting server'' message
during Unison startup (under the graphical UI).
\end{itemize}
\item Bug fixes:
\begin{itemize}
\item Fixed small bug in UI that neglected to change the displayed column 
  headers if loading a new profile caused the roots to change.
\item Fixed a bug that would put the text UI into an infinite loop if it
  encountered a conflict when run in batch mode.
\item Added some code to try to fix the display of non-Ascii characters in 
  filenames on Windows systems in the GTK UI.  (This code is currently 
  untested---if you're one of the people that had reported problems with
  display of non-ascii filenames, we'd appreciate knowing if this actually 
  fixes things.)
\item `\verb|-prefer/-force newer|' works properly now.  
        (The bug was reported by Sebastian Urbaniak and Sean Fulton.)
\end{itemize}
\item User interface and Unison behavior:
\begin{itemize}
\item Renamed `Proceed' to `Go' in the graphical UI.
\item Added exit status for the textual user interface.
\item Paths that are not synchronized because of conflicts or errors during 
  update detection are now noted in the log file.
\item \verb|[END]| messages in log now use a briefer format
\item Changed the text UI startup sequence so that
  {\tt ./unison -ui text} will use the default profile instead of failing.
\item Made some improvements to the error messages.
\item Added some debugging messages to remote.ml.
\end{itemize}
\end{changesfromversion}

\begin{changesfromversion}{2.7.7}
\item Incorporated, once again, a multi-threaded transport sub-system.
  It transfers several files at the same time, thereby making much
  more effective use of available network bandwidth.  Unlike the
  earlier attempt, this time we do not rely on the native thread
  library of OCaml.  Instead, we implement a light-weight,
  non-preemptive multi-thread library in OCaml directly.  This version
  appears stable.  

  Some adjustments to unison are made to accommodate the multi-threaded
  version.  These include, in particular, changes to the
  user interface and logging, for example:
  \begin{itemize}
  \item Two log entries for each transferring task, one for the
    beginning, one for the end.
  \item Suppressed warning messages against removing temp files left
    by a previous unison run, because warning does not work nicely
    under multi-threading.  The temp file names are made less likely
    to coincide with the name of a file created by the user.  They
    take the form \\ \verb|.#<filename>.<serial>.unison.tmp|.
    [N.b. This was later changed to \verb|.unison.<filename>.<serial>.unison.tmp|.]
  \end{itemize}
\item Added a new command to the GTK user interface: pressing 'f' causes
  Unison to start a new update detection phase, using as paths {\em just}
  those paths that have been detected as changed and not yet marked as
  successfully completed.  Use this command to quickly restart Unison on
  just the set of paths still needing attention after a previous run.
\item Made the {\tt ignorecase} preference user-visible, and changed the
  initialization code so that it can be manually set to true, even if
  neither host is running Windows.  (This may be useful, e.g., when using 
  Unison running on a Unix system with a FAT volume mounted.)
\item Small improvements and bug fixes:
  \begin{itemize}
  \item Errors in preference files now generate fatal errors rather than
    warnings at startup time.  (I.e., you can't go on from them.)  Also,
    we fixed a bug that was preventing these warnings from appearing in the
    text UI, so some users who have been running (unsuspectingly) with 
    garbage in their prefs files may now get error reports.
  \item Error reporting for preference files now provides file name and
    line number.
  \item More intelligible message in the case of identical change to the same 
    files: ``Nothing to do: replicas have been changed only in identical 
    ways since last sync.''
  \item Files with prefix '.\#' excluded when scanning for preference
    files.
  \item Rsync instructions are send directly instead of first
    marshaled.
  \item Won't try forever to get the fingerprint of a continuously changing file:
    unison will give up after certain number of retries.
  \item Other bug fixes, including the one reported by Peter Selinger
    (\verb|force=older preference| not working).
  \end{itemize}
\item Compilation:
  \begin{itemize}
  \item Upgraded to the new OCaml 3.04 compiler, with the LablGtk
    1.2.3 library (patched version used for compiling under Windows).
  \item Added the option to compile unison on the Windows platform with
    Cygwin GNU C compiler.  This option only supports building
    dynamically linked unison executables.
  \end{itemize}
\end{changesfromversion}

\begin{changesfromversion}{2.7.4}
\item Fixed a silly (but debilitating) bug in the client startup sequence.
\end{changesfromversion}

\begin{changesfromversion}{2.7.1}
\item Added \verb|addprefsto| preference, which (when set) controls which
preference file new preferences (e.g. new ignore patterns) are added to.
\item Bug fix: read the initial connection header one byte at a time, so
that we don't block if the header is shorter than expected.  (This bug
did not affect normal operation --- it just made it hard to tell when you
were trying to use Unison incorrectly with an old version of the server,
since it would hang instead of giving an error message.)
\end{changesfromversion}

\begin{changesfromversion}{2.6.59}
\item Changed \verb|fastcheck| from a boolean to a string preference.  Its 
  legal values are \verb|yes| (for a fast check), \verb|no| (for a safe 
  check), or \verb|default| (for a fast check---which also happens to be 
  safe---when running on Unix and a safe check when on Windows).  The default 
  is \verb|default|.
  \item Several preferences have been renamed for consistency.  All
  preference names are now spelled out in lowercase.  For backward
  compatibility, the old names still work, but they are not mentioned in
  the manual any more.
\item The temp files created by the 'diff' and 'merge' commands are now
   named by {\em pre}pending a new prefix to the file name, rather than
   appending a suffix.  This should avoid confusing diff/merge programs
   that depend on the suffix  to guess the type of the file contents.
\item We now set the keepalive option on the server socket, to make sure
  that the server times out if the communication link is unexpectedly broken. 
\item Bug fixes:
\begin{itemize}
\item When updating small files, Unison now closes the destination file.
\item File permissions are properly updated when the file is behind a
  followed link.
\item Several other small fixes.
\end{itemize}
\end{changesfromversion}


\begin{changesfromversion}{2.6.38}
\item Major Windows performance improvement!  

We've added a preference \verb|fastcheck| that makes Unison look only at
a file's creation time and last-modified time to check whether it has
changed.  This should result in a huge speedup when checking for updates
in large replicas.

  When this switch is set, Unison will use file creation times as 
  'pseudo inode numbers' when scanning Windows replicas for updates, 
  instead of reading the full contents of every file.  This may cause 
  Unison to miss propagating an update if the create time, 
  modification time, and length of the file are all unchanged by 
  the update (this is not easy to achieve, but it can be done).  
  However, Unison will never {\em overwrite} such an update with
  a change from the other replica, since it 
  always does a safe check for updates just before propagating a 
  change.  Thus, it is reasonable to use this switch most of the time 
  and occasionally run Unison once with {\tt fastcheck} set to false, 
  if you are worried that Unison may have overlooked an update.

  Warning: This change is has not yet been thoroughly field-tested.  If you 
  set the \verb|fastcheck| preference, pay careful attention to what
  Unison is doing.

\item New functionality: centralized backups and merging 
\begin{itemize}
\item This version incorporates two pieces of major new functionality,
   implemented by Sylvain Roy during a summer internship at Penn: a
   {\em centralized backup} facility that keeps a full backup of
   (selected files 
   in) each replica, and a {\em merging} feature that allows Unison to
   invoke an external file-merging tool to resolve conflicting changes to
   individual files.
 
\item Centralized backups:
\begin{itemize}
  \item Unison now maintains full backups of the last-synchronized versions
      of (some of) the files in each replica; these function both as
      backups in the usual sense
      and as the ``common version'' when invoking external
      merge programs.
  \item The backed up files are stored in a directory ~/.unison/backup on each
      host.  (The name of this directory can be changed by setting
      the environment variable \verb|UNISONBACKUPDIR|.)
  \item The predicate \verb|backup| controls which files are actually
     backed up:
      giving the preference '\verb|backup = Path *|' causes backing up
      of all files.
  \item Files are added to the backup directory whenever unison updates
      its archive.  This means that
      \begin{itemize}
       \item When unison reconstructs its archive from scratch (e.g., 
           because of an upgrade, or because the archive files have
           been manually deleted), all files will be backed up.
       \item Otherwise, each file will be backed up the first time unison
           propagates an update for it.
      \end{itemize}
  \item The preference \verb|backupversions| controls how many previous
      versions of each file are kept.  The default is 2 (i.e., the last 
      synchronized version plus one backup).
  \item For backward compatibility, the \verb|backups| preference is also
      still supported, but \verb|backup| is now preferred.
  \item It is OK to manually delete files from the backup directory (or to throw
      away the directory itself).  Before unison uses any of these files for 
      anything important, it checks that its fingerprint matches the one 
      that it expects. 
\end{itemize}

\item Merging:
\begin{itemize}
  \item Both user interfaces offer a new 'merge' command, invoked by pressing
      'm' (with a changed file selected).  
  \item The actual merging is performed by an external program.  
      The preferences \verb|merge| and \verb|merge2| control how this
      program is invoked.  If a backup exists for this file (see the
      \verb|backup| preference), then the \verb|merge| preference is used for 
      this purpose; otherwise \verb|merge2| is used.  In both cases, the 
      value of the preference should be a string representing the command 
      that should be passed to a shell to invoke the 
      merge program.  Within this string, the special substrings
      \verb|CURRENT1|, \verb|CURRENT2|, \verb|NEW|,  and \verb|OLD| may appear
      at any point.  Unison will substitute these as follows before invoking
      the command:
        \begin{itemize}
        \item \relax\verb|CURRENT1| is replaced by the name of the local 
        copy of the file;
        \item \relax\verb|CURRENT2| is replaced by the name of a temporary
        file, into which the contents of the remote copy of the file have
        been transferred by Unison prior to performing the merge;
        \item \relax\verb|NEW| is replaced by the name of a temporary
        file that Unison expects to be written by the merge program when
        it finishes, giving the desired new contents of the file; and
        \item \relax\verb|OLD| is replaced by the name of the backed up
        copy of the original version of the file (i.e., its state at the 
        end of the last successful run of Unison), if one exists 
        (applies only to \verb|merge|, not \verb|merge2|).
        \end{itemize}
      For example, on Unix systems setting the \verb|merge| preference to
\begin{verbatim}
   merge = diff3 -m CURRENT1 OLD CURRENT2 > NEW
\end{verbatim}
      will tell Unison to use the external \verb|diff3| program for merging.  

      A large number of external merging programs are available.  For 
      example, \verb|emacs| users may find the following convenient:
\begin{verbatim}
    merge2 = emacs -q --eval '(ediff-merge-files "CURRENT1" "CURRENT2" 
               nil "NEW")' 
    merge = emacs -q --eval '(ediff-merge-files-with-ancestor 
               "CURRENT1" "CURRENT2" "OLD" nil "NEW")' 
\end{verbatim}
(These commands are displayed here on two lines to avoid running off the
edge of the page.  In your preference file, each should be written on a
single line.) 

  \item If the external program exits without leaving any file at the
  path \verb|NEW|, 
      Unison considers the merge to have failed.  If the merge program writes
      a file called \verb|NEW| but exits with a non-zero status code,
      then Unison 
      considers the merge to have succeeded but to have generated conflicts.
      In this case, it attempts to invoke an external editor so that the
      user can resolve the conflicts.  The value of the \verb|editor| 
      preference controls what editor is invoked by Unison.  The default
      is \verb|emacs|.

  \item Please send us suggestions for other useful values of the
       \verb|merge2| and \verb|merge| preferences -- we'd like to give several 
       examples in the manual.
\end{itemize}
\end{itemize}

\item Smaller changes:
\begin{itemize}
\item When one preference file includes another, unison no longer adds the
  suffix '\verb|.prf|' to the included file by default.  If a file with 
  precisely the given name exists in the .unison directory, it will be used; 
  otherwise Unison will 
  add \verb|.prf|, as it did before.  (This change means that included 
  preference files can be named \verb|blah.include| instead of 
  \verb|blah.prf|, so that unison will not offer them in its 'choose 
  a preference file' dialog.)
\item For Linux systems, we now offer both a statically linked and a dynamically
  linked executable.  The static one is larger, but will probably run on more
  systems, since it doesn't depend on the same versions of dynamically
  linked library modules being available.
\item Fixed the \verb|force| and \verb|prefer| preferences, which were
  getting the propagation direction exactly backwards.
\item Fixed a bug in the startup code that would cause unison to crash
  when the default profile (\verb|~/.unison/default.prf|) does not exist.
\item Fixed a bug where, on the run when a profile is first created, 
  Unison would confusingly display the roots in reverse order in the user
  interface.
\end{itemize}

\item For developers:
\begin{itemize}
\item We've added a module dependency diagram to the source distribution, in
   \verb|src/DEPENDENCIES.ps|, to help new prospective developers with
   navigating the code. 
\end{itemize}
\end{changesfromversion}

\begin{changesfromversion}{2.6.11}
\item \incompatible{} Archive format has changed.  

\item \incompatible{} The startup sequence has been completely rewritten
and greatly simplified.  The main user-visible change is that the
\verb|defaultpath| preference has been removed.  Its effect can be
approximated by using multiple profiles, with \verb|include| directives
to incorporate common settings.  All uses of \verb|defaultpath| in
existing profiles should be changed to \verb|path|.

Another change in startup behavior that will affect some users is that it
is no longer possible to specify roots {\em both} in the profile {\em
  and} on the command line.

You can achieve a similar effect, though, by breaking your profile into
two:
\begin{verbatim}
  
  default.prf = 
      root = blah
      root = foo
      include common

  common.prf = 
      <everything else>
\end{verbatim}
Now do
\begin{verbatim}
  unison common root1 root2
\end{verbatim}
when you want to specify roots explicitly.

\item The \verb|-prefer| and \verb|-force| options have been extended to
allow users to specify that files with more recent modtimes should be
propagated, writing either \verb|-prefer newer| or \verb|-force newer|.
(For symmetry, Unison will also accept \verb|-prefer older| or
\verb|-force older|.)  The \verb|-force older/newer| options can only be
used when \verb|-times| is also set.

The graphical user interface provides access to these facilities on a
one-off basis via the \verb|Actions| menu.

\item Names of roots can now be ``aliased'' to allow replicas to be
relocated without changing the name of the archive file where Unison
stores information between runs.  (This feature is for experts only.  See
the ``Archive Files'' section of the manual for more information.)

\item Graphical user-interface:
\begin{itemize}
\item A new command is provided in the Synchronization menu for
  switching to a new profile without restarting Unison from scratch.
\item The GUI also supports one-key shortcuts for commonly
used profiles.  If a profile contains a preference of the form 
%
'\verb|key = n|', where \verb|n| is a single digit, then pressing this
key will cause Unison to immediately switch to this profile and begin
synchronization again from scratch.  (Any actions that may have been
selected for a set of changes currently being displayed will be
discarded.) 

\item Each profile may include a preference '\verb|label = <string>|' giving a
  descriptive string that described the options selected in this profile.
  The string is listed along with the profile name in the profile selection
  dialog, and displayed in the top-right corner of the main Unison window.
\end{itemize}

\item Minor:
\begin{itemize}
\item Fixed a bug that would sometimes cause the 'diff' display to order
  the files backwards relative to the main user interface.  (Thanks
  to Pascal Brisset for this fix.)
\item On Unix systems, the graphical version of Unison will check the
  \verb|DISPLAY| variable and, if it is not set, automatically fall back
  to the textual user interface.
\item Synchronization paths (\verb|path| preferences) are now matched
  against the ignore preferences.  So if a path is both specified in a
  \verb|path| preference and ignored, it will be skipped.
\item Numerous other bugfixes and small improvements.
\end{itemize}
\end{changesfromversion}

\begin{changesfromversion}{2.6.1}
\item The synchronization of modification times has been disabled for
  directories.

\item Preference files may now include lines of the form
  \verb+include <name>+, which will cause \verb+name.prf+ to be read
  at that point.

\item The synchronization of permission between Windows and Unix now
  works properly.

\item A binding \verb|CYGWIN=binmode| in now added to the environment
  so that the Cygwin port of OpenSSH works properly in a non-Cygwin
  context.

\item The \verb|servercmd| and \verb|addversionno| preferences can now
  be used together: \verb|-addversionno| appends an appropriate
  \verb+-NNN+ to the server command, which is found by using the value
  of the \verb|-servercmd| preference if there is one, or else just
  \verb|unison|.

\item Both \verb|'-pref=val'| and \verb|'-pref val'| are now allowed for
  boolean values.  (The former can be used to set a preference to false.)

\item Lot of small bugs fixed.
\end{changesfromversion}

\begin{changesfromversion}{2.5.31}
\item The \verb|log| preference is now set to \verb|true| by default,
  since the log file seems useful for most users.  
\item Several miscellaneous bugfixes (most involving symlinks).
\end{changesfromversion}

\begin{changesfromversion}{2.5.25}
\item \incompatible{} Archive format has changed (again).  

\item Several significant bugs introduced in 2.5.25 have been fixed.  
\end{changesfromversion}

\begin{changesfromversion}{2.5.1}
\item \incompatible{} Archive format has changed.  Make sure you
synchronize your replicas before upgrading, to avoid spurious
conflicts.  The first sync after upgrading will be slow.

\item New functionality:
\begin{itemize}
\item Unison now synchronizes file modtimes, user-ids, and group-ids.  

These new features are controlled by a set of new preferences, all of
which are currently \verb|false| by default.  

\begin{itemize}
\item When the \verb|times| preference is set to \verb|true|, file
modification times are propaged.  (Because the representations of time
may not have the same granularity on both replicas, Unison may not always
be able to make the modtimes precisely equal, but it will get them as
close as the operating systems involved allow.)
\item When the \verb|owner| preference is set to \verb|true|, file
ownership information is synchronized.
\item When the \verb|group| preference is set to \verb|true|, group 
information is synchronized.
\item When the \verb|numericIds| preference is set to \verb|true|, owner
and group information is synchronized numerically.  By default, owner and
group numbers are converted to names on each replica and these names are
synchronized.  (The special user id 0 and the special group 0 are never
mapped via user/group names even if this preference is not set.)
\end{itemize}

\item Added an integer-valued preference \verb|perms| that can be used to
control the propagation of permission bits.  The value of this preference
is a mask indicating which permission bits should be synchronized.  It is
set by default to $0o1777$: all bits but the set-uid and set-gid bits are
synchronised (synchronizing theses latter bits can be a security hazard).
If you want to synchronize all bits, you can set the value of this
preference to $-1$.

\item Added a \verb|log| preference (default \verb|false|), which makes
Unison keep a complete record of the changes it makes to the replicas.
By default, this record is written to a file called \verb|unison.log| in
the user's home directory (the value of the \verb|HOME| environment
variable).  If you want it someplace else, set the \verb|logfile|
preference to the full pathname you want Unison to use.

\item Added an \verb|ignorenot| preference that maintains a set of patterns 
  for paths that should definitely {\em not} be ignored, whether or not
  they match an \verb|ignore| pattern.  (That is, a path will now be ignored
  iff it matches an ignore pattern and does not match any ignorenot patterns.)
\end{itemize}
  
\item User-interface improvements:
\begin{itemize}
\item Roots are now displayed in the user interface in the same order
as they were given on the command line or in the preferences file.
\item When the \verb|batch| preference is set, the graphical user interface no 
  longer waits for user confirmation when it displays a warning message: it
  simply pops up an advisory window with a Dismiss button at the bottom and
  keeps on going.
\item Added a new preference for controlling how many status messages are
  printed during update detection: \verb|statusdepth| controls the maximum
  depth for paths on the local machine (longer paths are not displayed, nor
  are non-directory paths).  The value should be an integer; default is 1.  
\item Removed the \verb|trace| and \verb|silent| preferences.  They did
not seem very useful, and there were too many preferences for controlling
output in various ways.
\item The text UI now displays just the default command (the one that
will be used if the user just types \verb|<return>|) instead of all
available commands.  Typing \verb|?| will print the full list of
possibilities.
\item The function that finds the canonical hostname of the local host
(which is used, for example, in calculating the name of the archive file
used to remember which files have been synchronized) normally uses the
\verb|gethostname| operating system call.  However, if the environment
variable \verb|UNISONLOCALHOSTNAME| is set, its value will now be used
instead.  This makes it easier to use Unison in situations where a
machine's name changes frequently (e.g., because it is a laptop and gets
moved around a lot).
\item File owner and group are now displayed in the ``detail window'' at
the bottom of the screen, when unison is configured to synchronize them.
\end{itemize}

\item For hackers:
\begin{itemize}
\item Updated to Jacques Garrigue's new version of \verb|lablgtk|, which
  means we can throw away our local patched version.  

  If you're compiling the GTK version of unison from sources, you'll need
  to update your copy of lablgtk to the developers release.
  (Warning: installing lablgtk under Windows is currently a bit
  challenging.) 

\item The TODO.txt file (in the source distribution) has been cleaned up
and reorganized.  The list of pending tasks should be much easier to
make sense of, for people that may want to contribute their programming
energies.  There is also a separate file BUGS.txt for open bugs.
\item The Tk user interface has been removed (it was not being maintained
and no longer compiles).
\item The \verb|debug| preference now prints quite a bit of additional
information that should be useful for identifying sources of problems.
\item The version number of the remote server is now checked right away 
  during the connection setup handshake, rather than later.  (Somebody
  sent a bug report of a server crash that turned out to come from using
  inconsistent versions: better to check this earlier and in a way that
  can't crash either client or server.)
\item Unison now runs correctly on 64-bit architectures (e.g. Alpha
linux).  We will not be distributing binaries for these architectures
ourselves (at least for a while) but if someone would like to make them
available, we'll be glad to provide a link to them.
\end{itemize}

\item Bug fixes:
\begin{itemize}
\item Pattern matching (e.g. for \verb|ignore|) is now case-insensitive
  when Unison is in case-insensitive mode (i.e., when one of the replicas
  is on a windows machine).
\item Some people had trouble with mysterious failures during
  propagation of updates, where files would be falsely reported as having
  changed during synchronization.  This should be fixed.
\item Numerous smaller fixes.
\end{itemize}
\end{changesfromversion}

\begin{changesfromversion}{2.4.1}
\item Added a number of 'sorting modes' for the user interface.  By
default, conflicting changes are displayed at the top, and the rest of
the entries are sorted in alphabetical order.  This behavior can be
changed in the following ways:
\begin{itemize}
\item Setting  the \verb|sortnewfirst| preference to \verb|true| causes
newly created files to be displayed before changed files.
\item Setting \verb|sortbysize| causes files to be displayed in
increasing order of size.
\item Giving the preference \verb|sortfirst=<pattern>| (where
\verb|<pattern>| is a path descriptor in the same format as 'ignore' and 'follow'
patterns, causes paths matching this pattern to be displayed first.
\item Similarly, giving the preference \verb|sortlast=<pattern>| 
causes paths matching this pattern to be displayed last.
\end{itemize}
The sorting preferences are described in more detail in the user manual.
The \verb|sortnewfirst| and \verb|sortbysize| flags can also be accessed
from the 'Sort' menu in the grpahical user interface.

\item Added two new preferences that can be used to change unison's
fundamental behavior to make it more like a mirroring tool instead of
a synchronizer.
\begin{itemize}
\item Giving the preference \verb|prefer| with argument \verb|<root>|
(by adding \verb|-prefer <root>| to the command line or \verb|prefer=<root>|)
to your profile) means that, if there is a conflict, the contents of
\verb|<root>| 
should be propagated to the other replica (with no questions asked).
Non-conflicting changes are treated as usual.
\item Giving the preference \verb|force| with argument \verb|<root>|
will make unison resolve {\em all} differences in favor of the given
root, even if it was the other replica that was changed.
\end{itemize}
These options should be used with care!  (More information is available in
the manual.)

\item Small changes:
\begin{itemize}
\item 
Changed default answer to 'Yes' in all two-button dialogs in the 
  graphical interface (this seems more intuitive).

\item The \verb|rsync| preference has been removed (it was used to
activate rsync compression for file transfers, but rsync compression is
now enabled by default). 
\item  In the text user interface, the arrows indicating which direction
changes are being 
  propagated are printed differently when the user has overridded Unison's
  default recommendation (\verb|====>| instead of \verb|---->|).  This
  matches the behavior of the graphical interface, which displays such
  arrows in a different color.
\item Carriage returns (Control-M's) are ignored at the ends of lines in
  profiles, for Windows compatibility.
\item All preferences are now fully documented in the user manual. 
\end{itemize}
\end{changesfromversion}

\begin{changesfromversion}{2.3.12}
\item \incompatible{} Archive format has changed.  Make sure you
synchronize your replicas before upgrading, to avoid spurious
conflicts.  The first sync after upgrading will be slow.

\item New/improved functionality:
\begin{itemize}
\item  A new preference -sortbysize controls the order in which changes
  are displayed to the user: when it is set to true, the smallest
  changed files are displayed first.  (The default setting is false.) 
\item A new preference -sortnewfirst causes newly created files to be 
  listed before other updates in the user interface.
\item We now allow the ssh protocol to specify a port.  
\item Incompatible change: The unison: protocol is deprecated, and we added
  file: and socket:.  You may have to modify your profiles in the
  .unison directory.
  If a replica is specified without an explicit protocol, we now
  assume it refers to a file.  (Previously "//saul/foo" meant to use
  SSH to connect to saul, then access the foo directory.  Now it means
  to access saul via a remote file mechanism such as samba; the old
  effect is now achieved by writing {\tt ssh://saul/foo}.)
\item Changed the startup sequence for the case where roots are given but
  no profile is given on the command line.  The new behavior is to
  use the default profile (creating it if it does not exist), and
  temporarily override its roots.  The manual claimed that this case
  would work by reading no profile at all, but AFAIK this was never
  true.
\item In all user interfaces, files with conflicts are always listed first
\item A new preference 'sshversion' can be used to control which version
  of ssh should be used to connect to the server.  Legal values are 1 and 2.
  (Default is empty, which will make unison use whatever version of ssh
  is installed as the default 'ssh' command.)
\item The situation when the permissions of a file was updated the same on
  both side is now handled correctly (we used to report a spurious conflict)

\end{itemize}

\item Improvements for the Windows version:
\begin{itemize}
\item The fact that filenames are treated case-insensitively under
Windows should now be handled correctly.  The exact behavior is described
in the cross-platform section of the manual.
\item It should be possible to synchronize with Windows shares, e.g.,
  //host/drive/path.
\item Workarounds to the bug in syncing root directories in Windows.
The most difficult thing to fix is an ocaml bug: Unix.opendir fails on
c: in some versions of Windows.
\end{itemize}

\item Improvements to the GTK user interface (the Tk interface is no
longer being maintained): 
\begin{itemize}
\item The UI now displays actions differently (in blue) when they have been
  explicitly changed by the user from Unison's default recommendation.
\item More colorful appearance.
\item The initial profile selection window works better.
\item If any transfers failed, a message to this effect is displayed along with
  'Synchronization complete' at the end of the transfer phase (in case they
  may have scrolled off the top).
\item Added a global progress meter, displaying the percentage of {\em total}
  bytes that have been transferred so far.
\end{itemize}

\item Improvements to the text user interface:
\begin{itemize}
\item The file details will be displayed automatically when a
  conflict is been detected.
\item when a warning is generated (e.g. for a temporary
  file left over from a previous run of unison) Unison will no longer
  wait for a response if it is running in -batch mode.
\item The UI now displays a short list of possible inputs each time it waits
  for user interaction.  
\item The UI now quits immediately (rather than looping back and starting
  the interaction again) if the user presses 'q' when asked whether to 
  propagate changes.
\item Pressing 'g' in the text user interface will proceed immediately
  with propagating updates, without asking any more questions.
\end{itemize}

\item Documentation and installation changes:
\begin{itemize}
\item The manual now includes a FAQ, plus sections on common problems and
on tricks contributed by users.
\item Both the download page and the download directory explicitly say
what are the current stable and beta-test version numbers.
\item The OCaml sources for the up-to-the-minute developers' version (not
guaranteed to be stable, or even to compile, at any given time!) are now
available from the download page.
\item Added a subsection to the manual describing cross-platform
  issues (case conflicts, illegal filenames)
\end{itemize}

\item Many small bug fixes and random improvements.

\end{changesfromversion}

\begin{changesfromversion}{2.3.1}
\item Several bug fixes.  The most important is a bug in the rsync
module that would occasionally cause change propagation to fail with a
'rename' error.
\end{changesfromversion}

\begin{changesfromversion}{2.2}
\item The multi-threaded transport system is now disabled by default.
(It is not stable enough yet.)
\item Various bug fixes.
\item A new experimental feature: 

  The final component of a -path argument may now be the wildcard 
  specifier \verb|*|.  When Unison sees such a path, it expands this path on 
  the client into into the corresponding list of paths by listing the
  contents of that directory.  

  Note that if you use wildcard paths from the command line, you will
  probably need to use quotes or a backslash to prevent the * from
  being interpreted by your shell.

  If both roots are local, the contents of the first one will be used
  for expanding wildcard paths.  (Nb: this is the first one {\em after} the
  canonization step -- i.e., the one that is listed first in the user 
  interface -- not the one listed first on the command line or in the
  preferences file.)
\end{changesfromversion}

\begin{changesfromversion}{2.1}
\item The transport subsystem now includes an implementation by
Sylvain Gommier and Norman Ramsey of Tridgell and Mackerras's
\verb|rsync| protocol.  This protocol achieves much faster 
transfers when only a small part of a large file has been changed by
sending just diffs.  This feature is mainly helpful for transfers over
slow links---on fast local area networks it can actually degrade
performance---so we have left it off by default.  Start unison with
the \verb|-rsync| option (or put \verb|rsync=true| in your preferences
file) to turn it on.

\item ``Progress bars'' are now diplayed during remote file transfers,
showing what percentage of each file has been transferred so far.

\item The version numbering scheme has changed.  New releases will now
      be have numbers like 2.2.30, where the second component is
      incremented on every significant public release and the third
      component is the ``patch level.''

\item Miscellaneous improvements to the GTK-based user interface.
\item The manual  is now available in PDF format.

\item We are experimenting with using a multi-threaded transport
subsystem to transfer several files at the same time, making
much more effective use of available network bandwidth.  This feature
is not completely stable yet, so by default it is disabled in the
release version of Unison.

If you want to play with the multi-threaded version, you'll need to
recompile Unison from sources (as described in the documentation),
setting the THREADS flag in Makefile.OCaml to true.  Make sure that
your OCaml compiler has been installed with the \verb|-with-pthreads|
configuration option.  (You can verify this by checking whether the
file \verb|threads/threads.cma| in the OCaml standard library
directory contains the string \verb|-lpthread| near the end.)
\end{changesfromversion}

\begin{changesfromversion}{1.292}
\item Reduced memory footprint (this is especially important during
the first run of unison, where it has to gather information about all
the files in both repositories). 
\item Fixed a bug that would cause the socket server under NT to fail
  after the client exits. 
\item Added a SHIFT modifier to the Ignore menu shortcut keys in GTK
  interface (to avoid hitting them accidentally).  
\end{changesfromversion}

\begin{changesfromversion}{1.231}
\item Tunneling over ssh is now supported in the Windows version.  See
the installation section of the manual for detailed instructions.

\item The transport subsystem now includes an implementation of the
\verb|rsync| protocol, built by Sylvain Gommier and Norman Ramsey.
This protocol achieves much faster transfers when only a small part of
a large file has been changed by sending just diffs.  The rsync
feature is off by default in the current version.  Use the
\verb|-rsync| switch to turn it on.  (Nb.  We still have a lot of
tuning to do: you may not notice much speedup yet.)

\item We're experimenting with a multi-threaded transport subsystem,
written by Jerome Vouillon.  The downloadable binaries are still
single-threaded: if you want to try the multi-threaded version, you'll
need to recompile from sources.  (Say \verb|make THREADS=true|.)
Native thread support from the compiler is required.  Use the option
\verb|-threads N| to select the maximal number of concurrent 
threads (default is 5).  Multi-threaded
and single-threaded clients/servers can interoperate.  

\item A new GTK-based user interface is now available, thanks to
Jacques Garrigue.  The Tk user interface still works, but we'll be
shifting development effort to the GTK interface from now on.
\item OCaml 3.00 is now required for compiling Unison from sources.
The modules \verb|uitk| and \verb|myfileselect| have been changed to
use labltk instead of camltk.  To compile the Tk interface in Windows,
you must have ocaml-3.00 and tk8.3.  When installing tk8.3, put it in
\verb|c:\Tcl| rather than the suggested \verb|c:\Program Files\Tcl|, 
and be sure to install the headers and libraries (which are not 
installed by default).

\item Added a new \verb|-addversionno| switch, which causes unison to
use \verb|unison-<currentversionnumber>| instead of just \verb|unison|
as the remote server command.  This allows multiple versions of unison
to coexist conveniently on the same server: whichever version is run
on the client, the same version will be selected on the server.
\end{changesfromversion}

\begin{changesfromversion}{1.219}
\item \incompatible{} Archive format has changed.  Make sure you
synchronize your replicas before upgrading, to avoid spurious
conflicts.  The first sync after upgrading will be slow.

\item This version fixes several annoying bugs, including:
\begin{itemize}
\item Some cases where propagation of file permissions was not
working.
\item umask is now ignored when creating directories
\item directories are create writable, so that a read-only directory and
    its contents can be propagated.
\item Handling of warnings generated by the server.
\item Synchronizing a path whose parent is not a directory on both sides is
now flagged as erroneous.  
\item Fixed some bugs related to symnbolic links and nonexistant roots.
\begin{itemize}
\item 
   When a change (deletion or new contents) is propagated onto a 
     'follow'ed symlink, the file pointed to by the link is now changed.
     (We used to change the link itself, which doesn't fit our assertion
     that 'follow' means the link is completely invisible)
   \item When one root did not exist, propagating the other root on top of it
     used to fail, becuase unison could not calculate the working directory
     into which to write changes.  This should be fixed.
\end{itemize}
\end{itemize}

\item A human-readable timestamp has been added to Unison's archive files.

\item The semantics of Path and Name regular expressions now
correspond better. 

\item Some minor improvements to the text UI (e.g. a command for going
back to previous items)

\item The organization of the export directory has changed --- should
be easier to find / download things now.
\end{changesfromversion}

\begin{changesfromversion}{1.200}
\item \incompatible{} Archive format has changed.  Make sure you
synchronize your replicas before upgrading, to avoid spurious
conflicts.  The first sync after upgrading will be slow.

\item This version has not been tested extensively on Windows.

\item Major internal changes designed to make unison safer to run
at the same time as the replicas are being changed by the user.

\item Internal performance improvements.  
\end{changesfromversion}

\begin{changesfromversion}{1.190}
\item \incompatible{} Archive format has changed.  Make sure you
synchronize your replicas before upgrading, to avoid spurious
conflicts.  The first sync after upgrading will be slow.

\item A number of internal functions have been changed to reduce the
amount of memory allocation, especially during the first
synchronization.  This should help power users with very big replicas.

\item Reimplementation of low-level remote procedure call stuff, in
preparation for adding rsync-like smart file transfer in a later
release.   

\item Miscellaneous bug fixes.
\end{changesfromversion}

\begin{changesfromversion}{1.180}
\item \incompatible{} Archive format has changed.  Make sure you
synchronize your replicas before upgrading, to avoid spurious
conflicts.  The first sync after upgrading will be slow.

\item Fixed some small bugs in the interpretation of ignore patterns. 

\item Fixed some problems that were preventing the Windows version
from working correctly when click-started.

\item Fixes to treatment of file permissions under Windows, which were
causing spurious reports of different permissions when synchronizing
between windows and unix systems.

\item Fixed one more non-tail-recursive list processing function,
which was causing stack overflows when synchronizing very large
replicas. 
\end{changesfromversion}

\begin{changesfromversion}{1.169}
\item The text user interface now provides commands for ignoring
  files. 
\item We found and fixed some {\em more} non-tail-recursive list
  processing functions.  Some power users have reported success with
  very large replicas.
\item \incompatible 
Files ending in \verb|.tmp| are no longer ignored automatically.  If you want
to ignore such files, put an appropriate ignore pattern in your profile.

\item \incompatible{} The syntax of {\tt ignore} and {\tt follow}
patterns has changed. Instead of putting a line of the form
\begin{verbatim}
                 ignore = <regexp>
\end{verbatim}
  in your profile ({\tt .unison/default.prf}), you should put:
\begin{verbatim}
                 ignore = Regex <regexp>
\end{verbatim}
Moreover, two other styles of pattern are also recognized:
\begin{verbatim}
                 ignore = Name <name>
\end{verbatim}
matches any path in which one component matches \verb|<name>|, while
\begin{verbatim}
                 ignore = Path <path>
\end{verbatim}
matches exactly the path \verb|<path>|.

Standard ``globbing'' conventions can be used in \verb|<name>| and
\verb|<path>|:  
\begin{itemize}
\item a \verb|?| matches any single character except \verb|/|
\item a \verb|*| matches any sequence of characters not including \verb|/|
\item \verb|[xyz]| matches any character from the set $\{{\tt x},
  {\tt y}, {\tt z} \}$
\item \verb|{a,bb,ccc}| matches any one of \verb|a|, \verb|bb|, or
  \verb|ccc|. 
\end{itemize}

See the user manual for some examples.
\end{changesfromversion}

\begin{changesfromversion}{1.146}
\item Some users were reporting stack overflows when synchronizing
  huge directories.  We found and fixed some non-tail-recursive list
  processing functions, which we hope will solve the problem.  Please 
  give it a try and let us know.
\item Major additions to the documentation.  
\end{changesfromversion}

\begin{changesfromversion}{1.142}
\item Major internal tidying and many small bugfixes.
\item Major additions to the user manual.
\item Unison can now be started with no arguments -- it will prompt
automatically for the name of a profile file containing the roots to
be synchronized.  This makes it possible to start the graphical UI
from a desktop icon.
\item Fixed a small bug where the text UI on NT was raising a 'no such
  signal' exception.
\end{changesfromversion}

\begin{changesfromversion}{1.139}
\item The precompiled windows binary in the last release was compiled
with an old OCaml compiler, causing propagation of permissions not to
work (and perhaps leading to some other strange behaviors we've heard
reports about).  This has been corrected.  If you're using precompiled
binaries on Windows, please upgrade.
\item Added a \verb|-debug| command line flag, which controls debugging
of various modules.  Say \verb|-debug XXX| to enable debug tracing for
module \verb|XXX|, or \verb|-debug all| to turn on absolutely everything.
\item Fixed a small bug where the text UI on NT was raising a 'no such signal'
exception.
\end{changesfromversion}

\begin{changesfromversion}{1.111}
\item \incompatible{} The names and formats of the preference files in
the .unison directory have changed.  In particular:
\begin{itemize}
\item the file ``prefs'' should be renamed to default.prf
\item the contents of the file ``ignore'' should be merged into
  default.prf.  Each line of the form \verb|REGEXP| in ignore should
  become a line of the form \verb|ignore = REGEXP| in default.prf.
\end{itemize}
\item Unison now handles permission bits and  symbolic links.  See the
manual for details.

\item You can now have different preference files in your .unison
directory.  If you start unison like this
\begin{verbatim}
             unison profilename
\end{verbatim}
(i.e. with just one ``anonymous'' command-line argument), then the
file \verb|~/.unison/profilename.prf| will be loaded instead of
\verb|default.prf|. 

\item Some improvements to terminal handling in the text user interface

\item Added a switch -killServer that terminates the remote server process
when the unison client is shutting down, even when using sockets for 
communication.  (By default, a remote server created using ssh/rsh is 
terminated automatically, while a socket server is left running.)
\item When started in 'socket server' mode, unison prints 'server started' on
  stderr when it is ready to accept connections.  
  (This may be useful for scripts that want to tell when a socket-mode server 
  has finished initalization.)
\item We now make a nightly mirror of our current internal development
  tree, in case anyone wants an up-to-the-minute version to hack
  around with.
\item Added a file CONTRIB with some suggestions for how to help us
make Unison better.
\end{changesfromversion}

